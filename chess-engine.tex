% chess-engine.tex -- A chess engine in pure pdfLaTeX
% No Lua, no external programs for logic.
% Uses \count registers 200-263 for the 8x8 board.
% Piece encoding: 0=empty, 1=wP, 2=wN, 3=wB, 4=wR, 5=wQ, 6=wK
%                 negatives for black (-1=bP, -2=bN, etc.)

\makeatletter

%% ========================================================
%% Step 1: Board Representation & Coordinates
%% ========================================================

% Square index: (rank-1)*8 + file, where file a=1..h=8, rank 1..8
% So a1=1, b1=2, ..., h1=8, a2=9, ..., h8=64
% Stored in \count registers 200..263 (square index i -> \count\numexpr199+i\relax)

% Set a square value: \setsq{index}{value}
\def\setsq#1#2{\global\count\numexpr199+#1\relax=#2\relax}

% Get a square value (expandable): \getsq{index}
\def\getsq#1{\the\count\numexpr199+#1\relax}

% Precompute file and rank lookup tables for all 64 squares.
% \numexpr division ROUNDS (not truncates), so we use \divide (which truncates).
\count190=1\relax
\loop\ifnum\count190<65
  \count191=\count190\relax \advance\count191 by -1\relax % 0-based index
  \count192=\count191\relax \divide\count192 by 8\relax   % rank 0-based (truncating)
  \count193=\count192\relax \multiply\count193 by 8\relax
  \count194=\count191\relax \advance\count194 by -\count193\relax % file 0-based
  \advance\count192 by 1\relax % rank 1-based
  \advance\count194 by 1\relax % file 1-based
  \expandafter\edef\csname sq@file@\the\count190\endcsname{\the\count194}%
  \expandafter\edef\csname sq@rank@\the\count190\endcsname{\the\count192}%
  \advance\count190 by 1\relax
\repeat

% Extract file (1-8) from square index (expandable via csname lookup)
\def\sqfile#1{\csname sq@file@#1\endcsname}

% Extract rank (1-8) from square index (expandable via csname lookup)
\def\sqrank#1{\csname sq@rank@#1\endcsname}

% Compute square index from file and rank (no division, safe with \numexpr)
\def\sqindex#1#2{\the\numexpr(#2-1)*8+#1\relax}

% File letter <-> number lookups
\expandafter\def\csname filetonum@a\endcsname{1}
\expandafter\def\csname filetonum@b\endcsname{2}
\expandafter\def\csname filetonum@c\endcsname{3}
\expandafter\def\csname filetonum@d\endcsname{4}
\expandafter\def\csname filetonum@e\endcsname{5}
\expandafter\def\csname filetonum@f\endcsname{6}
\expandafter\def\csname filetonum@g\endcsname{7}
\expandafter\def\csname filetonum@h\endcsname{8}

\expandafter\def\csname numtofile@1\endcsname{a}
\expandafter\def\csname numtofile@2\endcsname{b}
\expandafter\def\csname numtofile@3\endcsname{c}
\expandafter\def\csname numtofile@4\endcsname{d}
\expandafter\def\csname numtofile@5\endcsname{e}
\expandafter\def\csname numtofile@6\endcsname{f}
\expandafter\def\csname numtofile@7\endcsname{g}
\expandafter\def\csname numtofile@8\endcsname{h}

\def\filetonum#1{\csname filetonum@#1\endcsname}
\def\numtofile#1{\csname numtofile@#1\endcsname}

% Helper: compute file and rank of a count register value into two count registers.
% Usage: \computefilerank{\the\somecount}{\mg@fromfile}{\mg@fromrank}
% Uses \count188 and \count189 as scratch.
\def\computefilerank#1#2#3{%
  \count188=#1\relax \advance\count188 by -1\relax
  \count189=\count188\relax \divide\count189 by 8\relax
  #3=\count189\relax \advance#3 by 1\relax
  \count189=\count189\relax \multiply\count189 by 8\relax
  \advance\count188 by -\count189\relax
  #2=\count188\relax \advance#2 by 1\relax
}

% Initialize board to starting position
\def\initboard{%
  % Clear all squares
  \count190=1\relax
  \loop\ifnum\count190<65
    \setsq{\the\count190}{0}%
    \advance\count190 by 1
  \repeat
  % White pieces (rank 1)
  \setsq{1}{4}%  a1 = wR
  \setsq{2}{2}%  b1 = wN
  \setsq{3}{3}%  c1 = wB
  \setsq{4}{5}%  d1 = wQ
  \setsq{5}{6}%  e1 = wK
  \setsq{6}{3}%  f1 = wB
  \setsq{7}{2}%  g1 = wN
  \setsq{8}{4}%  h1 = wR
  % White pawns (rank 2)
  \setsq{9}{1}\setsq{10}{1}\setsq{11}{1}\setsq{12}{1}%
  \setsq{13}{1}\setsq{14}{1}\setsq{15}{1}\setsq{16}{1}%
  % Black pawns (rank 7)
  \setsq{49}{-1}\setsq{50}{-1}\setsq{51}{-1}\setsq{52}{-1}%
  \setsq{53}{-1}\setsq{54}{-1}\setsq{55}{-1}\setsq{56}{-1}%
  % Black pieces (rank 8)
  \setsq{57}{-4}% a8 = bR
  \setsq{58}{-2}% b8 = bN
  \setsq{59}{-3}% c8 = bB
  \setsq{60}{-5}% d8 = bQ
  \setsq{61}{-6}% e8 = bK
  \setsq{62}{-3}% f8 = bB
  \setsq{63}{-2}% g8 = bN
  \setsq{64}{-4}% h8 = bR
}

%% ========================================================
%% Step 2: Game State
%% ========================================================

\newcount\sidetomove    % 1=white, -1=black
\newcount\castleWK      % White kingside castling right (0 or 1)
\newcount\castleWQ      % White queenside castling right
\newcount\castleBK      % Black kingside castling right
\newcount\castleBQ      % Black queenside castling right
\newcount\epsquare      % En passant target square (0=none)
\newcount\halfmoveclock % Half-move clock for 50-move rule
\newcount\fullmovenumber
\newcount\gameresult    % 0=ongoing, 1=white wins, -1=black wins, 2=draw

\newtoks\movehistory    % Token list for move history display

\newcount\gamehash      % Running hash for deterministic RNG seeding

% User-configurable game seed. Change this to get a different game.
% Override with \def\gameseed{...} before \initgamestate.
\def\gameseed{1}

\def\initgamestate{%
  \global\sidetomove=1\relax
  \global\castleWK=1\relax
  \global\castleWQ=1\relax
  \global\castleBK=1\relax
  \global\castleBQ=1\relax
  \global\epsquare=0\relax
  \global\halfmoveclock=0\relax
  \global\fullmovenumber=1\relax
  \global\gameresult=0\relax
  \global\movehistory={}%
  \global\gamehash=\gameseed\relax
}

% Update the hash with a user move and seed the RNG before the engine responds.
% Hash formula: h = (h * 7 + from + to * 64) mod 999979
% This ensures: same user moves -> same hash -> same engine replies across recompiles.
\def\seedbeforeenginemove{%
  {%
    \count186=\gamehash\relax
    \multiply\count186 by 7\relax
    \advance\count186 by \userfrom\relax
    \count187=\userto\relax
    \multiply\count187 by 64\relax
    \advance\count186 by \count187\relax
    % Modulo to prevent overflow (999979 is prime)
    \count187=\count186\relax
    \divide\count187 by 999979\relax
    \multiply\count187 by 999979\relax
    \advance\count186 by -\count187\relax
    \ifnum\count186<0 \advance\count186 by 999979\relax\fi
    \global\gamehash=\count186\relax
  }%
  \pdfsetrandomseed\gamehash\relax
}

%% ========================================================
%% Step 3: FEN Generation & Board Display
%% ========================================================

\def\piecetofenchar#1{%
  \ifcase\numexpr#1+6\relax
    k\or q\or r\or b\or n\or p\or ?\or P\or N\or B\or R\or Q\or K\fi
}

\newcount\fenrank
\newcount\fenfile
\newcount\fenempty
\newcount\fenpiece
\newcount\fensqidx

\def\generatefen{%
  \def\fenstring{}%
  \fenrank=8\relax
  \fen@rankloop
}

\def\fen@rankloop{%
  \ifnum\fenrank>0
    \fenempty=0\relax
    \fenfile=1\relax
    \fen@fileloop
    % Flush trailing empties
    \ifnum\fenempty>0
      \edef\fenstring{\fenstring\the\fenempty}%
    \fi
    \ifnum\fenrank>1
      \edef\fenstring{\fenstring/}%
    \fi
    \advance\fenrank by -1\relax
    \expandafter\fen@rankloop
  \fi
}

\def\fen@fileloop{%
  \ifnum\fenfile<9
    \fensqidx=\numexpr(\fenrank-1)*8+\fenfile\relax
    \fenpiece=\count\numexpr199+\fensqidx\relax
    \ifnum\fenpiece=0
      \advance\fenempty by 1\relax
    \else
      \ifnum\fenempty>0
        \edef\fenstring{\fenstring\the\fenempty}%
        \fenempty=0\relax
      \fi
      \edef\fenstring{\fenstring\piecetofenchar{\the\fenpiece}}%
    \fi
    \advance\fenfile by 1\relax
    \expandafter\fen@fileloop
  \fi
}

% Declare \ifnocastle outside any macro (must be allocated once)
\newif\ifnocastle

% Full FEN with game state
\def\generatefullfen{%
  \generatefen
  % Side to move
  \ifnum\sidetomove=1
    \edef\fenstring{\fenstring\space w}%
  \else
    \edef\fenstring{\fenstring\space b}%
  \fi
  % Castling
  \edef\fenstring{\fenstring\space}%
  \nocastletrue
  \ifnum\castleWK=1 \edef\fenstring{\fenstring K}\nocastlefalse\fi
  \ifnum\castleWQ=1 \edef\fenstring{\fenstring Q}\nocastlefalse\fi
  \ifnum\castleBK=1 \edef\fenstring{\fenstring k}\nocastlefalse\fi
  \ifnum\castleBQ=1 \edef\fenstring{\fenstring q}\nocastlefalse\fi
  \ifnocastle\edef\fenstring{\fenstring -}\fi
  % En passant
  \ifnum\epsquare=0
    \edef\fenstring{\fenstring\space -}%
  \else
    \edef\fenstring{\fenstring\space
      \numtofile{\sqfile{\the\epsquare}}\sqrank{\the\epsquare}}%
  \fi
  % Clocks
  \edef\fenstring{\fenstring\space\the\halfmoveclock\space\the\fullmovenumber}%
}

% Show board using chessboard package
\def\showboard{%
  \generatefen
  \edef\tmp@showboard{\noexpand\setchessboard{setfen=\fenstring}}%
  \tmp@showboard
  \chessboard
}

%% ========================================================
%% Step 4: Attack Detection
%% ========================================================

\newcount\attackresult
\newcount\atk@sq
\newcount\atk@color
\newcount\atk@file
\newcount\atk@rank
\newcount\atk@tf
\newcount\atk@tr
\newcount\atk@tsq
\newcount\atk@piece

% \issquareattacked{square}{attackercolor} -> sets \attackresult
\def\issquareattacked#1#2{%
  \attackresult=0\relax
  \atk@sq=#1\relax
  \atk@color=#2\relax
  \computefilerank{\the\atk@sq}{\atk@file}{\atk@rank}%
  % Check pawn attacks
  \ifnum\attackresult=0 \atk@checkpawns\fi
  % Check knight attacks
  \ifnum\attackresult=0 \atk@checkknights\fi
  % Check king attacks
  \ifnum\attackresult=0 \atk@checkking\fi
  % Check sliding attacks (rook/queen on orthogonals, bishop/queen on diagonals)
  \ifnum\attackresult=0 \atk@checksliding\fi
}

% Check pawn attacks on \atk@sq by \atk@color
\def\atk@checkpawns{%
  % Pawns of color c attack from rank (sq_rank - c) at file +/- 1
  \atk@tr=\numexpr\atk@rank-\atk@color\relax
  \ifnum\atk@tr>0 \ifnum\atk@tr<9
    % Left diagonal
    \atk@tf=\numexpr\atk@file-1\relax
    \ifnum\atk@tf>0
      \atk@tsq=\numexpr(\atk@tr-1)*8+\atk@tf\relax
      \atk@piece=\count\numexpr199+\atk@tsq\relax
      \ifnum\atk@piece=\numexpr\atk@color\relax
        \attackresult=1\relax
      \fi
    \fi
    % Right diagonal
    \ifnum\attackresult=0
      \atk@tf=\numexpr\atk@file+1\relax
      \ifnum\atk@tf<9
        \atk@tsq=\numexpr(\atk@tr-1)*8+\atk@tf\relax
        \atk@piece=\count\numexpr199+\atk@tsq\relax
        \ifnum\atk@piece=\numexpr\atk@color\relax
          \attackresult=1\relax
        \fi
      \fi
    \fi
  \fi\fi
}

% Check knight attacks
\def\atk@checkknights{%
  \atk@tryknightsrc{-2}{-1}%
  \ifnum\attackresult=0 \atk@tryknightsrc{-2}{1}\fi
  \ifnum\attackresult=0 \atk@tryknightsrc{-1}{-2}\fi
  \ifnum\attackresult=0 \atk@tryknightsrc{-1}{2}\fi
  \ifnum\attackresult=0 \atk@tryknightsrc{1}{-2}\fi
  \ifnum\attackresult=0 \atk@tryknightsrc{1}{2}\fi
  \ifnum\attackresult=0 \atk@tryknightsrc{2}{-1}\fi
  \ifnum\attackresult=0 \atk@tryknightsrc{2}{1}\fi
}

\def\atk@tryknightsrc#1#2{%
  \atk@tf=\numexpr\atk@file+(#1)\relax
  \atk@tr=\numexpr\atk@rank+(#2)\relax
  \ifnum\atk@tf>0 \ifnum\atk@tf<9 \ifnum\atk@tr>0 \ifnum\atk@tr<9
    \atk@tsq=\numexpr(\atk@tr-1)*8+\atk@tf\relax
    \atk@piece=\count\numexpr199+\atk@tsq\relax
    \ifnum\atk@piece=\numexpr2*\atk@color\relax
      \attackresult=1\relax
    \fi
  \fi\fi\fi\fi
}

% Check king attacks (adjacent squares)
\def\atk@checkking{%
  \atk@trykingsrc{-1}{-1}%
  \ifnum\attackresult=0 \atk@trykingsrc{-1}{0}\fi
  \ifnum\attackresult=0 \atk@trykingsrc{-1}{1}\fi
  \ifnum\attackresult=0 \atk@trykingsrc{0}{-1}\fi
  \ifnum\attackresult=0 \atk@trykingsrc{0}{1}\fi
  \ifnum\attackresult=0 \atk@trykingsrc{1}{-1}\fi
  \ifnum\attackresult=0 \atk@trykingsrc{1}{0}\fi
  \ifnum\attackresult=0 \atk@trykingsrc{1}{1}\fi
}

\def\atk@trykingsrc#1#2{%
  \atk@tf=\numexpr\atk@file+(#1)\relax
  \atk@tr=\numexpr\atk@rank+(#2)\relax
  \ifnum\atk@tf>0 \ifnum\atk@tf<9 \ifnum\atk@tr>0 \ifnum\atk@tr<9
    \atk@tsq=\numexpr(\atk@tr-1)*8+\atk@tf\relax
    \atk@piece=\count\numexpr199+\atk@tsq\relax
    \ifnum\atk@piece=\numexpr6*\atk@color\relax
      \attackresult=1\relax
    \fi
  \fi\fi\fi\fi
}

% Check sliding piece attacks (bishop, rook, queen)
\def\atk@checksliding{%
  % Orthogonal rays: check for rook (4) or queen (5)
  \atk@castray{0}{1}{4}{5}%
  \ifnum\attackresult=0 \atk@castray{0}{-1}{4}{5}\fi
  \ifnum\attackresult=0 \atk@castray{1}{0}{4}{5}\fi
  \ifnum\attackresult=0 \atk@castray{-1}{0}{4}{5}\fi
  % Diagonal rays: check for bishop (3) or queen (5)
  \ifnum\attackresult=0 \atk@castray{1}{1}{3}{5}\fi
  \ifnum\attackresult=0 \atk@castray{1}{-1}{3}{5}\fi
  \ifnum\attackresult=0 \atk@castray{-1}{1}{3}{5}\fi
  \ifnum\attackresult=0 \atk@castray{-1}{-1}{3}{5}\fi
}

\newcount\ray@f
\newcount\ray@r
\newcount\ray@sq
\newcount\ray@p

\def\atk@castray#1#2#3#4{%
  \ray@f=\atk@file
  \ray@r=\atk@rank
  \def\ray@df{#1}\def\ray@dr{#2}%
  \def\ray@ta{#3}\def\ray@tb{#4}%
  \atk@rayloop
}

\def\atk@rayloop{%
  \advance\ray@f by \ray@df\relax
  \advance\ray@r by \ray@dr\relax
  \ifnum\ray@f>0 \ifnum\ray@f<9 \ifnum\ray@r>0 \ifnum\ray@r<9
    \ray@sq=\numexpr(\ray@r-1)*8+\ray@f\relax
    \ray@p=\count\numexpr199+\ray@sq\relax
    \ifnum\ray@p=0
      \expandafter\expandafter\expandafter\expandafter
      \expandafter\expandafter\expandafter\atk@rayloop
    \else
      % Hit a piece - check if it's the right attacker
      \ifnum\ray@p=\numexpr\ray@ta*\atk@color\relax
        \attackresult=1\relax
      \else\ifnum\ray@p=\numexpr\ray@tb*\atk@color\relax
        \attackresult=1\relax
      \fi\fi
    \fi
  \fi\fi\fi\fi
}

%% ========================================================
%% Step 5-8: Move Generation
%% ========================================================

\newcount\movecnt
\newtoks\movelist

\def\clearmovelist{%
  \global\movecnt=0\relax
  \global\movelist={}%
}

% Add a move: \addmove{from}{to}{flags}
% flags: 0=normal, 1=en passant, 2=castling, 3=promotion (queen)
\def\addmove#1#2#3{%
  \global\advance\movecnt by 1\relax
  \begingroup
    \edef\tmp@add{%
      \global\noexpand\movelist={%
        \the\movelist\noexpand\moveentry{#1}{#2}{#3}}}%
    \tmp@add
  \endgroup
}

% --- Knight move generation ---

\newcount\mg@from
\newcount\mg@fromfile
\newcount\mg@fromrank
\newcount\mg@tf
\newcount\mg@tr
\newcount\mg@tsq
\newcount\mg@tpiece

\def\genknightmoves#1{%
  \mg@from=#1\relax
  \computefilerank{\the\mg@from}{\mg@fromfile}{\mg@fromrank}%
  \tryknightjump{-2}{-1}%
  \tryknightjump{-2}{1}%
  \tryknightjump{-1}{-2}%
  \tryknightjump{-1}{2}%
  \tryknightjump{1}{-2}%
  \tryknightjump{1}{2}%
  \tryknightjump{2}{-1}%
  \tryknightjump{2}{1}%
}

\def\tryknightjump#1#2{%
  \mg@tf=\numexpr\mg@fromfile+(#1)\relax
  \mg@tr=\numexpr\mg@fromrank+(#2)\relax
  \ifnum\mg@tf>0 \ifnum\mg@tf<9 \ifnum\mg@tr>0 \ifnum\mg@tr<9
    \mg@tsq=\numexpr(\mg@tr-1)*8+\mg@tf\relax
    \mg@tpiece=\count\numexpr199+\mg@tsq\relax
    % Can move if empty or enemy piece
    \ifnum\sidetomove>0
      \ifnum\mg@tpiece<1 % empty or black piece
        \addmove{\the\mg@from}{\the\mg@tsq}{0}%
      \fi
    \else
      \ifnum\mg@tpiece>-1 % empty or white piece
        \addmove{\the\mg@from}{\the\mg@tsq}{0}%
      \fi
    \fi
  \fi\fi\fi\fi
}

% --- King move generation (non-castling) ---

\def\genkingmoves#1{%
  \mg@from=#1\relax
  \computefilerank{\the\mg@from}{\mg@fromfile}{\mg@fromrank}%
  \trykingstep{-1}{-1}%
  \trykingstep{-1}{0}%
  \trykingstep{-1}{1}%
  \trykingstep{0}{-1}%
  \trykingstep{0}{1}%
  \trykingstep{1}{-1}%
  \trykingstep{1}{0}%
  \trykingstep{1}{1}%
  % Castling
  \gencastlingmoves
}

\def\trykingstep#1#2{%
  \mg@tf=\numexpr\mg@fromfile+(#1)\relax
  \mg@tr=\numexpr\mg@fromrank+(#2)\relax
  \ifnum\mg@tf>0 \ifnum\mg@tf<9 \ifnum\mg@tr>0 \ifnum\mg@tr<9
    \mg@tsq=\numexpr(\mg@tr-1)*8+\mg@tf\relax
    \mg@tpiece=\count\numexpr199+\mg@tsq\relax
    \ifnum\sidetomove>0
      \ifnum\mg@tpiece<1
        \addmove{\the\mg@from}{\the\mg@tsq}{0}%
      \fi
    \else
      \ifnum\mg@tpiece>-1
        \addmove{\the\mg@from}{\the\mg@tsq}{0}%
      \fi
    \fi
  \fi\fi\fi\fi
}

% --- Castling move generation ---

\def\gencastlingmoves{%
  \ifnum\sidetomove>0
    % White kingside: e1(5)->g1(7), f1(6) and g1(7) must be empty
    \ifnum\castleWK=1
      \ifnum\mg@from=5
        \mg@tpiece=\count\numexpr199+6\relax
        \ifnum\mg@tpiece=0
          \mg@tpiece=\count\numexpr199+7\relax
          \ifnum\mg@tpiece=0
            \addmove{5}{7}{2}%
          \fi
        \fi
      \fi
    \fi
    % White queenside: e1(5)->c1(3), b1(2),c1(3),d1(4) must be empty
    \ifnum\castleWQ=1
      \ifnum\mg@from=5
        \mg@tpiece=\count\numexpr199+4\relax
        \ifnum\mg@tpiece=0
          \mg@tpiece=\count\numexpr199+3\relax
          \ifnum\mg@tpiece=0
            \mg@tpiece=\count\numexpr199+2\relax
            \ifnum\mg@tpiece=0
              \addmove{5}{3}{2}%
            \fi
          \fi
        \fi
      \fi
    \fi
  \else
    % Black kingside: e8(61)->g8(63), f8(62) and g8(63) must be empty
    \ifnum\castleBK=1
      \ifnum\mg@from=61
        \mg@tpiece=\count\numexpr199+62\relax
        \ifnum\mg@tpiece=0
          \mg@tpiece=\count\numexpr199+63\relax
          \ifnum\mg@tpiece=0
            \addmove{61}{63}{2}%
          \fi
        \fi
      \fi
    \fi
    % Black queenside: e8(61)->c8(59), b8(58),c8(59),d8(60) must be empty
    \ifnum\castleBQ=1
      \ifnum\mg@from=61
        \mg@tpiece=\count\numexpr199+60\relax
        \ifnum\mg@tpiece=0
          \mg@tpiece=\count\numexpr199+59\relax
          \ifnum\mg@tpiece=0
            \mg@tpiece=\count\numexpr199+58\relax
            \ifnum\mg@tpiece=0
              \addmove{61}{59}{2}%
            \fi
          \fi
        \fi
      \fi
    \fi
  \fi
}

% --- Sliding piece move generation ---

\newcount\sl@f
\newcount\sl@r
\newcount\sl@sq
\newcount\sl@p

\def\genslidingmoves#1#2{%
  \mg@from=#1\relax
  \computefilerank{\the\mg@from}{\mg@fromfile}{\mg@fromrank}%
  \ifnum#2=1
    \castray{1}{1}\castray{1}{-1}\castray{-1}{1}\castray{-1}{-1}%
  \fi
  \ifnum#2=2
    \castray{1}{0}\castray{-1}{0}\castray{0}{1}\castray{0}{-1}%
  \fi
  \ifnum#2=3
    \castray{1}{1}\castray{1}{-1}\castray{-1}{1}\castray{-1}{-1}%
    \castray{1}{0}\castray{-1}{0}\castray{0}{1}\castray{0}{-1}%
  \fi
}

\def\castray#1#2{%
  \sl@f=\mg@fromfile
  \sl@r=\mg@fromrank
  \def\sl@df{#1}\def\sl@dr{#2}%
  \castrayloop
}

\def\castrayloop{%
  \advance\sl@f by \sl@df\relax
  \advance\sl@r by \sl@dr\relax
  \ifnum\sl@f>0 \ifnum\sl@f<9 \ifnum\sl@r>0 \ifnum\sl@r<9
    \sl@sq=\numexpr(\sl@r-1)*8+\sl@f\relax
    \sl@p=\count\numexpr199+\sl@sq\relax
    \ifnum\sl@p=0
      \addmove{\the\mg@from}{\the\sl@sq}{0}%
      \expandafter\expandafter\expandafter\expandafter
      \expandafter\expandafter\expandafter\castrayloop
    \else
      \ifnum\sidetomove>0
        \ifnum\sl@p<0
          \addmove{\the\mg@from}{\the\sl@sq}{0}%
        \fi
      \else
        \ifnum\sl@p>0
          \addmove{\the\mg@from}{\the\sl@sq}{0}%
        \fi
      \fi
    \fi
  \fi\fi\fi\fi
}

% --- Pawn move generation ---

\newcount\pawn@dir
\newcount\pawn@startrank
\newcount\pawn@promrank
\newcount\pawn@nextsq
\newcount\pawn@nextp
\newcount\pawn@nextrank

\def\genpawnmoves#1{%
  \mg@from=#1\relax
  \computefilerank{\the\mg@from}{\mg@fromfile}{\mg@fromrank}%
  \ifnum\sidetomove>0
    \pawn@dir=1\relax
    \pawn@startrank=2\relax
    \pawn@promrank=8\relax
  \else
    \pawn@dir=-1\relax
    \pawn@startrank=7\relax
    \pawn@promrank=1\relax
  \fi
  % Single push
  \pawn@nextrank=\numexpr\mg@fromrank+\pawn@dir\relax
  \ifnum\pawn@nextrank>0 \ifnum\pawn@nextrank<9
    \pawn@nextsq=\numexpr(\pawn@nextrank-1)*8+\mg@fromfile\relax
    \pawn@nextp=\count\numexpr199+\pawn@nextsq\relax
    \ifnum\pawn@nextp=0
      \ifnum\pawn@nextrank=\pawn@promrank
        \addmove{\the\mg@from}{\the\pawn@nextsq}{3}%
      \else
        \addmove{\the\mg@from}{\the\pawn@nextsq}{0}%
      \fi
      % Double push from starting rank
      \ifnum\mg@fromrank=\pawn@startrank
        \pawn@nextrank=\numexpr\mg@fromrank+2*\pawn@dir\relax
        \pawn@nextsq=\numexpr(\pawn@nextrank-1)*8+\mg@fromfile\relax
        \pawn@nextp=\count\numexpr199+\pawn@nextsq\relax
        \ifnum\pawn@nextp=0
          \addmove{\the\mg@from}{\the\pawn@nextsq}{0}%
        \fi
      \fi
    \fi
  \fi\fi
  % Captures
  \trypawncapture{-1}%
  \trypawncapture{1}%
}

\def\trypawncapture#1{%
  \mg@tf=\numexpr\mg@fromfile+(#1)\relax
  \pawn@nextrank=\numexpr\mg@fromrank+\pawn@dir\relax
  \ifnum\mg@tf>0 \ifnum\mg@tf<9 \ifnum\pawn@nextrank>0 \ifnum\pawn@nextrank<9
    \pawn@nextsq=\numexpr(\pawn@nextrank-1)*8+\mg@tf\relax
    \pawn@nextp=\count\numexpr199+\pawn@nextsq\relax
    % Normal capture: enemy piece present
    \ifnum\sidetomove>0
      \ifnum\pawn@nextp<0
        \ifnum\pawn@nextrank=\pawn@promrank
          \addmove{\the\mg@from}{\the\pawn@nextsq}{3}%
        \else
          \addmove{\the\mg@from}{\the\pawn@nextsq}{0}%
        \fi
      \fi
    \else
      \ifnum\pawn@nextp>0
        \ifnum\pawn@nextrank=\pawn@promrank
          \addmove{\the\mg@from}{\the\pawn@nextsq}{3}%
        \else
          \addmove{\the\mg@from}{\the\pawn@nextsq}{0}%
        \fi
      \fi
    \fi
    % En passant capture
    \ifnum\pawn@nextsq=\epsquare
      \addmove{\the\mg@from}{\the\pawn@nextsq}{1}%
    \fi
  \fi\fi\fi\fi
}

%% ========================================================
%% Step 9: Generate All Pseudo-Legal Moves & Legal Filtering
%% ========================================================

\newcount\gen@sq
\newcount\gen@piece
\newcount\gen@abspiece

\def\generateallmoves{%
  \clearmovelist
  \gen@sq=1\relax
  \gen@sqloop
}

\def\gen@sqloop{%
  \ifnum\gen@sq<65
    \gen@piece=\count\numexpr199+\gen@sq\relax
    \ifnum\gen@piece=0
    \else
      \ifnum\sidetomove>0
        \ifnum\gen@piece>0
          \gen@dispatch
        \fi
      \else
        \ifnum\gen@piece<0
          \gen@dispatch
        \fi
      \fi
    \fi
    \advance\gen@sq by 1\relax
    \expandafter\gen@sqloop
  \fi
}

\def\gen@dispatch{%
  \gen@abspiece=\gen@piece
  \ifnum\gen@abspiece<0 \gen@abspiece=-\gen@abspiece \fi
  \ifcase\gen@abspiece
    % 0 = empty
  \or % 1 = pawn
    \genpawnmoves{\the\gen@sq}%
  \or % 2 = knight
    \genknightmoves{\the\gen@sq}%
  \or % 3 = bishop
    \genslidingmoves{\the\gen@sq}{1}%
  \or % 4 = rook
    \genslidingmoves{\the\gen@sq}{2}%
  \or % 5 = queen
    \genslidingmoves{\the\gen@sq}{3}%
  \or % 6 = king
    \genkingmoves{\the\gen@sq}%
  \fi
}

% --- Make / Unmake Move ---

\newcount\mk@from
\newcount\mk@to
\newcount\mk@flags
\newcount\mk@piece
\newcount\mk@captured
\newcount\mk@tofile
\newcount\mk@fromrank

% Saved state for unmake
\newcount\save@captured
\newcount\save@castleWK
\newcount\save@castleWQ
\newcount\save@castleBK
\newcount\save@castleBQ
\newcount\save@epsquare
\newcount\save@halfmoveclock

\def\makemove#1#2#3{%
  \mk@from=#1\relax
  \mk@to=#2\relax
  \mk@flags=#3\relax
  \mk@piece=\count\numexpr199+\mk@from\relax
  \mk@captured=\count\numexpr199+\mk@to\relax
  % Save state for unmake
  \global\save@captured=\mk@captured
  \global\save@castleWK=\castleWK
  \global\save@castleWQ=\castleWQ
  \global\save@castleBK=\castleBK
  \global\save@castleBQ=\castleBQ
  \global\save@epsquare=\epsquare
  \global\save@halfmoveclock=\halfmoveclock
  % Move piece
  \setsq{\the\mk@to}{\the\mk@piece}%
  \setsq{\the\mk@from}{0}%
  % Handle special moves
  \ifnum\mk@flags=1 % en passant
    % Remove the captured pawn (same rank as from, same file as to)
    \computefilerank{\the\mk@to}{\mk@tofile}{\mk@fromrank}%
    % mk@tofile now has the file of the to-square
    % We need the rank of the from-square
    \mk@fromrank=\sqrank{\the\mk@from}\relax
    \setsq{\the\numexpr(\mk@fromrank-1)*8+\mk@tofile\relax}{0}%
  \fi
  \ifnum\mk@flags=2 % castling
    \ifnum\mk@to>\mk@from % kingside
      \setsq{\the\numexpr\mk@from+1\relax}{\the\count\numexpr199+\mk@to+1\relax}%
      \setsq{\the\numexpr\mk@to+1\relax}{0}%
    \else % queenside
      \setsq{\the\numexpr\mk@from-1\relax}{\the\count\numexpr199+\mk@to-2\relax}%
      \setsq{\the\numexpr\mk@to-2\relax}{0}%
    \fi
  \fi
  \ifnum\mk@flags=3 % promotion (to queen)
    \setsq{\the\mk@to}{\the\numexpr5*\sidetomove\relax}%
  \fi
}

\def\unmakemove#1#2#3{%
  \mk@from=#1\relax
  \mk@to=#2\relax
  \mk@flags=#3\relax
  \mk@piece=\count\numexpr199+\mk@to\relax
  % Handle promotion: restore to pawn
  \ifnum\mk@flags=3
    \mk@piece=\sidetomove
  \fi
  % Restore piece to origin
  \setsq{\the\mk@from}{\the\mk@piece}%
  % Restore captured piece (or empty)
  \setsq{\the\mk@to}{\the\save@captured}%
  % Handle en passant: restore captured pawn
  \ifnum\mk@flags=1
    \computefilerank{\the\mk@to}{\mk@tofile}{\mk@fromrank}%
    \mk@fromrank=\sqrank{\the\mk@from}\relax
    \setsq{\the\numexpr(\mk@fromrank-1)*8+\mk@tofile\relax}{\the\numexpr-\sidetomove\relax}%
  \fi
  % Handle castling: restore rook
  \ifnum\mk@flags=2
    \ifnum\mk@to>\mk@from % kingside
      \setsq{\the\numexpr\mk@to+1\relax}{\the\count\numexpr199+\mk@from+1\relax}%
      \setsq{\the\numexpr\mk@from+1\relax}{0}%
    \else % queenside
      \setsq{\the\numexpr\mk@to-2\relax}{\the\count\numexpr199+\mk@from-1\relax}%
      \setsq{\the\numexpr\mk@from-1\relax}{0}%
    \fi
  \fi
  % Restore game state
  \global\castleWK=\save@castleWK
  \global\castleWQ=\save@castleWQ
  \global\castleBK=\save@castleBK
  \global\castleBQ=\save@castleBQ
  \global\epsquare=\save@epsquare
  \global\halfmoveclock=\save@halfmoveclock
}

% --- Find king square ---

\newcount\fk@sq
\newcount\fk@target
\newcount\kingsquare

\def\findkingsquare#1{%
  \fk@target=\numexpr6*(#1)\relax
  \kingsquare=0\relax
  \fk@sq=1\relax
  \fk@loop
}

\def\fk@loop{%
  \ifnum\fk@sq<65
    \ifnum\count\numexpr199+\fk@sq\relax=\fk@target
      \kingsquare=\fk@sq
    \else
      \advance\fk@sq by 1\relax
      \expandafter\expandafter\expandafter\fk@loop
    \fi
  \fi
}

% --- Legal move filtering ---

\newcount\legalmovecnt
\newtoks\legalmovelist

\def\clearlegalmovelist{%
  \global\legalmovecnt=0\relax
  \global\legalmovelist={}%
}

\def\addlegalmove#1#2#3{%
  \global\advance\legalmovecnt by 1\relax
  \begingroup
    \edef\tmp@add{%
      \global\noexpand\legalmovelist={%
        \the\legalmovelist\noexpand\moveentry{#1}{#2}{#3}}}%
    \tmp@add
  \endgroup
}

\newcount\fl@from
\newcount\fl@to
\newcount\fl@flags
\newcount\fl@oppcolor

\def\filterlegal{%
  \clearlegalmovelist
  \fl@oppcolor=\numexpr-\sidetomove\relax
  \def\moveentry##1##2##3{%
    \fl@from=##1\relax
    \fl@to=##2\relax
    \fl@flags=##3\relax
    \fl@testmove{##1}{##2}{##3}%
  }%
  \the\movelist
}

\newcount\fl@legal

\def\fl@testmove#1#2#3{%
  \fl@legal=1\relax
  \ifnum\fl@flags=2
    \fl@testcastling{#1}{#2}{#3}%
  \else
    \makemove{#1}{#2}{#3}%
    \findkingsquare{\the\sidetomove}%
    \issquareattacked{\the\kingsquare}{\the\fl@oppcolor}%
    \ifnum\attackresult=1
      \fl@legal=0\relax
    \fi
    \unmakemove{#1}{#2}{#3}%
  \fi
  \ifnum\fl@legal=1
    \addlegalmove{#1}{#2}{#3}%
  \fi
}

\newcount\fl@transit

\def\fl@testcastling#1#2#3{%
  % King must not be in check now
  \findkingsquare{\the\sidetomove}%
  \issquareattacked{\the\kingsquare}{\the\fl@oppcolor}%
  \ifnum\attackresult=1
    \fl@legal=0\relax
  \fi
  \ifnum\fl@legal=1
    % Transit square must not be attacked
    \ifnum#2>#1
      \fl@transit=\numexpr#1+1\relax
    \else
      \fl@transit=\numexpr#1-1\relax
    \fi
    \issquareattacked{\the\fl@transit}{\the\fl@oppcolor}%
    \ifnum\attackresult=1
      \fl@legal=0\relax
    \fi
  \fi
  \ifnum\fl@legal=1
    \makemove{#1}{#2}{#3}%
    \issquareattacked{#2}{\the\fl@oppcolor}%
    \ifnum\attackresult=1
      \fl@legal=0\relax
    \fi
    \unmakemove{#1}{#2}{#3}%
  \fi
}

% Convenience: generate and filter in one call
\def\generatelegal{%
  \generateallmoves
  \filterlegal
}

%% ========================================================
%% Step 10: Game State Update (after move execution)
%% ========================================================

\def\updategamestate#1#2#3{%
  % Toggle side to move
  \global\sidetomove=\numexpr-\sidetomove\relax
  % Update castling rights
  \ifnum#1=5 \global\castleWK=0\relax\global\castleWQ=0\relax\fi
  \ifnum#1=61 \global\castleBK=0\relax\global\castleBQ=0\relax\fi
  \ifnum#1=1 \global\castleWQ=0\relax\fi
  \ifnum#1=8 \global\castleWK=0\relax\fi
  \ifnum#1=57 \global\castleBQ=0\relax\fi
  \ifnum#1=64 \global\castleBK=0\relax\fi
  \ifnum#2=1 \global\castleWQ=0\relax\fi
  \ifnum#2=8 \global\castleWK=0\relax\fi
  \ifnum#2=57 \global\castleBQ=0\relax\fi
  \ifnum#2=64 \global\castleBK=0\relax\fi
  % En passant square
  \global\epsquare=0\relax
  {%
    \count190=\count\numexpr199+(#2)\relax
    \count191=\numexpr#2-#1\relax
    \ifnum\count190=1
      \ifnum\count191=16
        \global\epsquare=\numexpr#1+8\relax
      \fi
    \fi
    \ifnum\count190=-1
      \ifnum\count191=-16
        \global\epsquare=\numexpr#1-8\relax
      \fi
    \fi
  }%
  % Half-move clock: reset on pawn move, promotion, en passant, or capture
  {%
    \count190=0 % flag: 1 = reset clock
    \count191=\count\numexpr199+(#2)\relax
    \ifnum\count191=1 \count190=1 \fi
    \ifnum\count191=-1 \count190=1 \fi
    \ifnum#3=3 \count190=1 \fi % promotion
    \ifnum#3=1 \count190=1 \fi % en passant
    \ifnum\save@captured=0 \else \count190=1 \fi % capture
    \ifnum\count190=1
      \global\halfmoveclock=0\relax
    \else
      \global\advance\halfmoveclock by 1\relax
    \fi
  }%
  % Full move number: increment after black moves
  \ifnum\sidetomove>0
    \global\advance\fullmovenumber by 1\relax
  \fi
}

%% ========================================================
%% Step 11: Random Move Selection
%% ========================================================

\newcount\randomindex
\newcount\pickcounter
\def\selectedfrom{}
\def\selectedto{}
\def\selectedflags{}

\def\pickrandommove{%
  \ifnum\legalmovecnt>0
    \randomindex=\numexpr\pdfuniformdeviate\legalmovecnt+1\relax
    \pickcounter=0\relax
    \def\moveentry##1##2##3{%
      \advance\pickcounter by 1\relax
      \ifnum\pickcounter=\randomindex
        \gdef\selectedfrom{##1}%
        \gdef\selectedto{##2}%
        \gdef\selectedflags{##3}%
      \fi
    }%
    \the\legalmovelist
  \fi
}

%% ========================================================
%% Step 12: User Move Parsing & Validation
%% ========================================================

\newcount\userfrom
\newcount\userto
\newcount\userfromfile
\newcount\userfromrank
\newcount\usertofile
\newcount\usertorank
\newcount\userflags
\newcount\promotiontarget  % 5=Q, 4=R, 3=B, 2=N (for underpromotion support)

% Promotion piece value lookup
\expandafter\def\csname promoval@q\endcsname{5}
\expandafter\def\csname promoval@r\endcsname{4}
\expandafter\def\csname promoval@b\endcsname{3}
\expandafter\def\csname promoval@n\endcsname{2}

\def\parsemove#1{%
  \promotiontarget=5\relax
  \parsemove@i#1\parsemove@end
}

\def\parsemove@i#1#2#3#4#5\parsemove@end{%
  \def\tmp@test{#3}%
  \def\tmp@dash{-}%
  \ifx\tmp@test\tmp@dash
    \parsemove@dash#4#5\parsemove@end{#1}{#2}%
  \else
    \userfromfile=\filetonum{#1}\relax
    \userfromrank=#2\relax
    \usertofile=\filetonum{#3}\relax
    \usertorank=#4\relax
    \userfrom=\numexpr(\userfromrank-1)*8+\userfromfile\relax
    \userto=\numexpr(\usertorank-1)*8+\usertofile\relax
    \parsemove@handlepromo#5\parsemove@promoend
  \fi
}

\def\parsemove@dash#1#2\parsemove@end#3#4{%
  \userfromfile=\filetonum{#3}\relax
  \userfromrank=#4\relax
  \usertofile=\filetonum{#1}\relax
  \usertorank=#2\relax
  \userfrom=\numexpr(\userfromrank-1)*8+\userfromfile\relax
  \userto=\numexpr(\usertorank-1)*8+\usertofile\relax
}

% Handle optional promotion suffix (q/r/b/n) from 5th character of move
\def\parsemove@handlepromo#1\parsemove@promoend{%
  \def\pm@tmp{#1}%
  \ifx\pm@tmp\@empty\else
    \promotiontarget=\csname promoval@#1\endcsname\relax
  \fi
}

\newcount\validresult
\newcount\validflags

\def\validateusermove{%
  \validresult=0\relax
  \def\moveentry##1##2##3{%
    \ifnum##1=\userfrom
      \ifnum##2=\userto
        \validresult=1\relax
        \validflags=##3\relax
      \fi
    \fi
  }%
  \the\legalmovelist
}

%% ========================================================
%% Step 13: Game Flow & End Detection
%% ========================================================

\newcount\gameover

\def\checkgameend{%
  \gameover=0\relax
  \generatelegal
  \ifnum\legalmovecnt=0
    \findkingsquare{\the\sidetomove}%
    \issquareattacked{\the\kingsquare}{\the\numexpr-\sidetomove\relax}%
    \ifnum\attackresult=1
      \global\gameresult=\numexpr-\sidetomove\relax
      \gameover=1\relax
    \else
      \global\gameresult=2\relax
      \gameover=2\relax
    \fi
  \fi
  \ifnum\halfmoveclock>99
    \global\gameresult=2\relax
    \gameover=2\relax
  \fi
}

% Format a move as coordinate notation for display
\def\fmtmove#1#2{%
  \numtofile{\sqfile{#1}}\sqrank{#1}%
  \numtofile{\sqfile{#2}}\sqrank{#2}%
}

% Execute a user move and have engine respond
\def\playmove#1{%
  \ifnum\gameresult=0
    \parsemove{#1}%
    \generatelegal
    \validateusermove
    \ifnum\validresult=0
      \message{ILLEGAL MOVE: #1}%
      \errmessage{Illegal move: #1. Aborting.}%
    \else
      \makemove{\the\userfrom}{\the\userto}{\the\validflags}%
      \updategamestate{\the\userfrom}{\the\userto}{\the\validflags}%
      % Record in history
      \edef\tmp@hist{\the\movehistory}%
      \ifnum\sidetomove<0
        \edef\tmp@hist{\tmp@hist\the\fullmovenumber. #1 }%
      \else
        \edef\tmp@hist{\tmp@hist #1\space\space}%
      \fi
      \global\movehistory=\expandafter{\tmp@hist}%
      \checkgameend
      \ifnum\gameover>0
        \displaygameresult
      \else
        \seedbeforeenginemove
        \enginemove
      \fi
    \fi
  \fi
}

\def\enginemove{%
  \generatelegal
  \ifnum\legalmovecnt>0
    \pickrandommove
    \makemove{\selectedfrom}{\selectedto}{\selectedflags}%
    \updategamestate{\selectedfrom}{\selectedto}{\selectedflags}%
    \edef\tmp@hist{\the\movehistory}%
    \edef\tmp@engmv{\fmtmove{\selectedfrom}{\selectedto}}%
    \ifnum\sidetomove<0
      \edef\tmp@hist{\tmp@hist\the\fullmovenumber. \tmp@engmv\space}%
    \else
      \edef\tmp@hist{\tmp@hist\tmp@engmv\space\space}%
    \fi
    \global\movehistory=\expandafter{\tmp@hist}%
    \message{Engine plays: \tmp@engmv}%
    \checkgameend
    \ifnum\gameover>0
      \displaygameresult
    \fi
  \fi
}

\def\displaygameresult{%
  \ifnum\gameresult=1
    \message{Game over: White wins by checkmate!}%
  \fi
  \ifnum\gameresult=-1
    \message{Game over: Black wins by checkmate!}%
  \fi
  \ifnum\gameresult=2
    \message{Game over: Draw!}%
  \fi
}

\def\showmovehistory{%
  \par\medskip\noindent\textbf{Moves:} \the\movehistory
  \ifnum\gameresult=1 \textbf{1-0}\fi
  \ifnum\gameresult=-1 \textbf{0-1}\fi
  \ifnum\gameresult=2 \textbf{1/2-1/2}\fi
  \par
}

%% ========================================================
%% Step 14: Output for Interactive Mode
%% ========================================================

\newwrite\engineout

\def\piecetoascii#1{%
  \ifcase\numexpr#1+6\relax
    k\or q\or r\or b\or n\or p\or .\or P\or N\or B\or R\or Q\or K\fi
}

\newcount\asc@rank
\newcount\asc@file
\newcount\asc@sq
\newcount\asc@p

\def\writeengineoutput{%
  \immediate\openout\engineout=engine-output.dat\relax
  \immediate\write\engineout{ENGINEMOVE:\tmp@engmv}%
  \ifnum\gameresult=0
    \immediate\write\engineout{STATUS:ongoing}%
  \fi
  \ifnum\gameresult=1
    \immediate\write\engineout{STATUS:white_wins}%
  \fi
  \ifnum\gameresult=-1
    \immediate\write\engineout{STATUS:black_wins}%
  \fi
  \ifnum\gameresult=2
    \immediate\write\engineout{STATUS:draw}%
  \fi
  \generatefullfen
  \immediate\write\engineout{FEN:\fenstring}%
  \immediate\write\engineout{BOARD:}%
  \asc@rank=8\relax
  \asc@writeranks
  \immediate\write\engineout{~~a~b~c~d~e~f~g~h}%
  \immediate\closeout\engineout
}

\def\asc@writeranks{%
  \ifnum\asc@rank>0
    \def\asc@line{}%
    \asc@file=1\relax
    \asc@writefiles
    \immediate\write\engineout{\the\asc@rank\space\asc@line}%
    \advance\asc@rank by -1\relax
    \expandafter\asc@writeranks
  \fi
}

\def\asc@writefiles{%
  \ifnum\asc@file<9
    \asc@sq=\numexpr(\asc@rank-1)*8+\asc@file\relax
    \asc@p=\count\numexpr199+\asc@sq\relax
    \edef\asc@line{\asc@line\space\piecetoascii{\the\asc@p}}%
    \advance\asc@file by 1\relax
    \expandafter\asc@writefiles
  \fi
}

\def\enginemoveinteractive{%
  \generatelegal
  \ifnum\legalmovecnt>0
    \pickrandommove
    \makemove{\selectedfrom}{\selectedto}{\selectedflags}%
    \updategamestate{\selectedfrom}{\selectedto}{\selectedflags}%
    \edef\tmp@engmv{\fmtmove{\selectedfrom}{\selectedto}}%
    \edef\tmp@hist{\the\movehistory}%
    \ifnum\sidetomove<0
      \edef\tmp@hist{\tmp@hist\the\fullmovenumber. \tmp@engmv\space}%
    \else
      \edef\tmp@hist{\tmp@hist\tmp@engmv\space\space}%
    \fi
    \global\movehistory=\expandafter{\tmp@hist}%
    \checkgameend
    \writeengineoutput
  \else
    \edef\tmp@engmv{none}%
    \writeengineoutput
  \fi
}

\def\playmovealiased#1{%
  \ifnum\gameresult=0
    \parsemove{#1}%
    \generatelegal
    \validateusermove
    \ifnum\validresult=0
      \immediate\openout\engineout=engine-output.dat\relax
      \immediate\write\engineout{ERROR:Illegal move: #1}%
      \immediate\closeout\engineout
    \else
      \makemove{\the\userfrom}{\the\userto}{\the\validflags}%
      \updategamestate{\the\userfrom}{\the\userto}{\the\validflags}%
      \edef\tmp@hist{\the\movehistory}%
      \ifnum\sidetomove<0
        \edef\tmp@hist{\tmp@hist\the\fullmovenumber. #1 }%
      \else
        \edef\tmp@hist{\tmp@hist #1\space\space}%
      \fi
      \global\movehistory=\expandafter{\tmp@hist}%
      \checkgameend
      \ifnum\gameover>0
        \edef\tmp@engmv{none}%
        \writeengineoutput
      \else
        \seedbeforeenginemove
        \enginemoveinteractive
      \fi
    \fi
  \fi
}

%% ========================================================
%% Step 15: UCI Support — Replay Moves & Engine Response
%% ========================================================

% Replay a single move (either color, no engine response).
% Used by the UCI wrapper to replay the position before generating a response.
\def\replaymove#1{%
  \ifnum\gameresult=0
    \parsemove{#1}%
    \generatelegal
    \validateusermove
    \ifnum\validresult=1
      \edef\replay@side{\the\sidetomove}%
      \makemove{\the\userfrom}{\the\userto}{\the\validflags}%
      \updategamestate{\the\userfrom}{\the\userto}{\the\validflags}%
      % Underpromotion fixup: if promotion to non-queen piece
      \ifnum\validflags=3
        \ifnum\promotiontarget<5
          \setsq{\the\userto}{\the\numexpr\promotiontarget*\replay@side\relax}%
        \fi
      \fi
      % Update hash for RNG continuity
      \seedbeforeenginemove
      % Record in history
      \edef\tmp@hist{\the\movehistory}%
      \ifnum\sidetomove<0
        \edef\tmp@hist{\tmp@hist\the\fullmovenumber. #1 }%
      \else
        \edef\tmp@hist{\tmp@hist #1\space\space}%
      \fi
      \global\movehistory=\expandafter{\tmp@hist}%
      \checkgameend
    \fi
  \fi
}

% Generate one engine move and write it to engine-output.dat.
% Used by the UCI wrapper after replaying all prior moves.
\def\generateengineresponse{%
  \ifnum\gameresult=0
    \pdfsetrandomseed\gamehash\relax
    \generatelegal
    \ifnum\legalmovecnt>0
      \pickrandommove
      \makemove{\selectedfrom}{\selectedto}{\selectedflags}%
      \updategamestate{\selectedfrom}{\selectedto}{\selectedflags}%
      \edef\tmp@engmv{\fmtmove{\selectedfrom}{\selectedto}}%
      % Append promotion suffix for UCI format
      \ifnum\selectedflags=3 \edef\tmp@engmv{\tmp@engmv q}\fi
      \checkgameend
    \else
      \checkgameend
      \edef\tmp@engmv{none}%
    \fi
    \writeengineoutput
  \fi
}

\makeatother
\endinput
